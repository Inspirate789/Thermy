\section*{Введение}
\addcontentsline{toc}{section}{Введение}

Стремительное развитие и внедрение технологий искусственного интеллекта и технологий автоматической обработки текстовой информации способствуют развитию лингвистических баз данных как основы создания прикладных программных средств, проведения лингвистических исследований источников информации при решении ряда прикладных задач, где однозначная и упорядоченная терминология имеет особую значимость. Под терминологической базой данных принято понимать организованную в соответствии с определёнными правилами и поддерживаемую в памяти компьютера совокупность данных, характеризующую актуальное состояние некоторой предметной области и используемую для удовлетворения информационных потребностей пользователей \cite{intro_1}. Каждый термин дополнен информацией о его значении, эквивалентах в других языках, кратких формах, синонимах, сведениях об области применения. По целевому назначению терминологические базы данных разделяют на одноязычные, предназначенные для обеспечения информацией о стандартизованной и рекомендованной терминологии, и многоязычные, ориентированные на работы по переводу научно-технической литературы и документации \cite{intro_2} \cite{intro_3}.

Создание терминологических баз данных представляет собой сложный и трудоемкий процесс, требующий значительного количества времени на их создание и обновление, что особенно важно для развивающихся терминологий таких предметных областей, как авиация, космонавтика, нанотехнологии, биоинженерия, информационные технологии и многих других. Одним из наиболее время-затратных процессов является ручной сбор иллюстративного материла -- извлечение специальной терминологии из коллекций текстов, что требует наличия средств автоматического извлечения многокомпонентных терминов при обработке научно-технических текстов.

Существующие программные средства автоматического извлечения терминов основаны на лингвистических и статистических методах. Могут быть использованы и методы машинного обучения \cite{intro_4}, сложность их реализации вызвана необходимостью наличия огромных массивов обучающих данных, которые могут отсутствовать для определенной предметной области. В основе лингвистических методов лежит использование грамматики лексико-синтаксических шаблонов, представляющих собой структурные модели лингвистических конструкций \cite{intro_5} \cite{intro_6}. Статистический подход заключается в нахождении n-грамм слов по заданным частотным характеристикам \cite{intro_7} \cite{intro_8}. Гибридный подход для выделения терминологических сочетаний, объединяющий лингвистический и статистический методы, заключается в предварительном описании моделей, по которым могут быть построены термины для последующего поиска их в коллекции текстов \cite{intro_9}.

Выравнивание терминологических единиц в параллельных текстах обычно осуществляется в два этапа: сначала выделяют терминологические единицы на каждом языке отдельно, затем один из одноязычных списков терминов-кандидатов интерпретируется как язык источника, и для каждого термина-кан-\\дидата на языке источнике предлагаются потенциально эквивалентные термины в списке терминов-кандидатов на языке перевода \cite{intro_10}.

Целью данной работы является разработка системы извлечения многокомпонентных терминов и их переводных эквивалентов из параллельных науч-\\но-технических текстов. Для достижения поставленной цели необходимо решить следующие задачи.

\begin{enumerate}[label*=\arabic*.]
	\item Провести анализ предметной области и формализовать задачу.
	\item Спроектировать базу данных и структуру программного обеспечения.
	\item Реализовать интерфейс для доступа к базе данных. 
	\item Реализовать программное обеспечение, которое позволит пользователю создавать, получать и изменять сведения из разработанной базы данных. 
	\item Провести исследование зависимости времени выполнения запросов от использования кеширования данных текущей сессии пользователя.
\end{enumerate}

\pagebreak