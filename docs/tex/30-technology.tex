\section{Технологическая часть}

% В данном разделе будут описаны критерии сравнения операционных систем для устройств интернета вещей.



\subsection{Анализ СУБД}

Взять в рассмотрение хотя бы 3 из них: 

\subsubsection{SQLite?}

\subsubsection{MySQL}

\subsubsection{Oracle}

\subsubsection{Microsoft SQL Server}

\subsubsection{PostgreSQL}

\subsubsection{MongoDB}

\subsubsection{Redis}

\subsubsection{Выбор СУБД для решения задачи}

PostgreSQL --- основное хранилище, Redis --- кеш данных текущей сессии (по окончании сессии перекидывается на диск на стороне сервера (после возврата из функции обработки подключения, т.е. на стороне сервера надо разделять (идентифицировать) конкретные сессии (не пользователей!))).



\subsection{Средства реализации}

Что на чём написано, как, куда и с помощью чего деплоилось (упомянуть Docker и контейнеризацию). Если в конечном варианте останется Envoy, то рассказать, зачем он нужен.

Golang + Python + SolidJS/React

Внешние средства для мониторинга состояния системы: Docker Desktop (или UI сервиса, на котором развёрнуто приложение), Grafana, PGAdmin.



\subsection{Детали реализации}

\subsubsection{Триггеры БД?}

\subsubsection{Роли БД}

\subsubsection{Примеры работы ПО}



\pagebreak