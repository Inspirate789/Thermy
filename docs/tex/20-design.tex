\section{Конструкторская часть}

% В данном разделе будет проведён анализ существующих операционных систем для устройств интернета вещей. Рассматриваемые операционные системы будут относиться к одному из двух типов: ОС реального времени и ОС разделения времени.

\subsection{Проектирование БД}

\subsubsection{Формализация сущностей системы}

\subsubsection{Триггеры БД?}

\subsubsection{Ролевая модель}



\subsection{Проектирование программного комплекса}

\subsubsection{Архитектура приложения}

\subsubsubsection{Монолитная архитектура}

\subsubsubsection{Микросервисная архитектура}

\subsubsubsection{Выбор архитектуры приложения}

Выбираем микросервисы, потому что очень сильно выигрываем от плюсов (особенно от разных языков, но пока не говорю, каких).



\subsubsection{Связь компонентов приложения}

\subsubsubsection{REST API}

\subsubsubsection{gRPC}

\subsubsubsection{Выбор взаимодействия компонентов приложения}

gRPC



\subsubsection{Структура программного комплекса}

На рисунке представлена структура программного комплекса, оформленная в виде диаграммы развёртывания. Она отражает компоненты, необходимые для работы системы, а также способы их взаимодействия. 



\subsubsection{Паттерны проектирования}

[Ссылка на книгу по паттернам]

Паттерны появились потому, что многие разработчики искали пути повышения гибкости и степени повторного использования своих программ.
Найденные решения воплощены в краткой и легко применимой на практике
форме. «Банда Четырех» объясняет каждый паттерн на простом примере
четким и понятным языком. Использование паттернов при разработке
программных систем позволяет проектировщику перейти на более высокий уровень разработки проекта. Теперь архитектор и программист могут
оперировать образными названиями паттернов и общаться на одном языке.

Паттерны проектирования упрощают повторное использование удачных
проектных и архитектурных решений. Представление прошедших проверку
временем методик в виде паттернов проектирования делает их более доступными для разработчиков новых систем. Паттерны проектирования помогают
выбрать альтернативные решения, упрощающие повторное использование
системы, и избежать тех альтернатив, которые его затрудняют. Паттерны
улучшают качество документации и сопровождения существующих систем, поскольку они позволяют явно описать взаимодействия классов и объектов,
а также причины, по которым система была построена так, а не иначе. Проще
говоря, паттерны проектирования дают разработчику возможность быстрее
найти правильный путь.

Далее будут описаны паттерны проектирования, которые будут использованы при разработке программного комплекса.

\subsubsubsection{Active Record}

\subsubsubsection{Dependency injection}

\subsubsubsection{Template method}

[Ссылка на книгу по паттернам]

Template Method (шаблонный метод) — паттерн поведения классов.

Шаблонный метод определяет скелет алгоритма, перекладывая ответственность за некоторые его шаги на подклассы. Это позволяет подклассам переопределять отдельные шаги алгоритма, не меняя его общей структуры.

Основные условия для применения паттерна шаблонный метод:
„- однократное использование инвариантных частей алгоритма, при этом реализация изменяющегося поведения остается на усмотрение подклассов;
„- необходимость вычленить и локализовать в одном классе поведение,
общее для всех подклассов, чтобы избежать дублирования кода. Это
хороший пример техники «вынесения за скобки с целью обобщения»,
описанной в работе Уильяма Опдайка (William Opdyke) и Ральфа
Джонсона (Ralph Johnson) [OJ93]. Сначала выявляются различия в существующем коде, которые затем выносятся в отдельные операции.
В конечном итоге различающиеся фрагменты кода заменяются шаблонным методом, из которого вызываются новые операции;
„- управление расширениями подклассов. Шаблонный метод можно определить так, что он будет вызывать операции-зацепки (hooks) — см. раздел «Результаты» — в определенных точках, разрешив тем самым расширение только в этих точках.

Шаблонные методы — один из фундаментальных приемов повторного
использования кода. Они играют особенно важную роль в библиотеках
классов, поскольку предоставляют возможность вынести общее поведение
в библиотечные классы.

Шаблонные методы приводят к инвертированной структуре кода, которая подразумевает, что родительский класс вызывает операции подкласса, а не наоборот.

\pagebreak