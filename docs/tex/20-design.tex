\section{Конструкторская часть}

% В данном разделе будет проведён анализ существующих операционных систем для устройств интернета вещей. Рассматриваемые операционные системы будут относиться к одному из двух типов: ОС реального времени и ОС разделения времени.

\subsection{Проектирование БД}

\subsubsection{Формализация сущностей системы}

Поле <<класс>> в моделях нужно для масштабирования.

\subsubsection{Триггеры БД?}

\subsubsection{Ролевая модель}



\subsection{Проектирование программного комплекса}

\subsubsection{Архитектура приложения}

\subsubsubsection{Монолитная архитектура}

\subsubsubsection{Микросервисная архитектура}

\subsubsubsection{Выбор архитектуры приложения}

Выбираем микросервисы, потому что очень сильно выигрываем от плюсов (особенно от разных языков, но пока не говорю, каких).



\subsubsection{Связь компонентов приложения}

\subsubsubsection{REST API}

\subsubsubsection{gRPC}

\subsubsubsection{Выбор взаимодействия компонентов приложения}

gRPC



\subsubsection{Структура программного комплекса}

На рисунке представлена структура программного комплекса, оформленная в виде диаграммы развёртывания. Она отражает компоненты, необходимые для работы системы, а также способы их взаимодействия. 



\subsubsection{Диаграммы последовательностей}



\subsubsection{Паттерны проектирования}

Паттерны проектирования повышают степень повторного использования проектных и архитектурных решений. Они помогают выбрать альтернативные решения, упрощающие повторное использование системы, и избежать тех альтернатив, которые его затрудняют. Паттерны улучшают качество документации и сопровождения существующих систем, поскольку они позволяют явно описать взаимодействия классов и объектов, а также причины, по которым система была построена так, а не иначе \cite{patterns}.

Далее будут описаны паттерны проектирования, которые будут использованы при разработке программного комплекса.

% \subsubsubsection{Active Record}

\subsubsubsection{Repository}

Репозиторий \cite{repository_pattern} --- это слой абстракции, инкапсулирующий в себе всё, что относится к способу хранения данных. Он предназначен для отделения бизнес-логики от деталей реализации слоя доступа к данным.

Паттерн Репозиторий стал популярным благодаря DDD (Domain Driven Design). В отличие от Database Driven Design, в DDD разработка начинается с проектирования бизнес логики, причём во внимание принимаются только особенности предметной области, а всё, что связано с особенностями хранения данных, игнорируется.

Применение данного паттерна не предполагает создание только одного объекта репозитория во всем приложении. Хорошей практикой считается создание отдельных репозиториев для каждого бизнес-объекта или контекста, например: OrdersRepository, UsersRepository, AdminRepository.

Репозиторий --- это высокоуровневая абстракция доступа к данным. Интерфейс каждого конкретного репозитория определяется в слое бизнес-логики наряду с классами предметной области. Реализация каждого репозитория находится в слое доступа к данным (Data Access Layer, DAL), который состоит из реализации каждого репозитория, ORM-специфичных классов, сущностей, классов-сопоставлений (mapping), контекстов данных и т.д.

\subsubsubsection{Dependency injection}

\subsubsubsection{Template method}

Template Method (шаблонный метод) \cite{patterns} --- паттерн поведения классов.

Шаблонный метод определяет скелет алгоритма, перекладывая ответственность за некоторые его шаги на подклассы. Это позволяет подклассам переопределять отдельные шаги алгоритма, не меняя его общей структуры.

Основные условия для применения паттерна шаблонный метод:

\begin{itemize}[label*=---]
	\item однократное использование инвариантных частей алгоритма, при этом реализация изменяющегося поведения остается на усмотрение подклассов;
	\item необходимость вычленить и локализовать в одном классе поведение, общее для всех подклассов, чтобы избежать дублирования кода;
	\item управление расширениями подклассов (шаблонный метод можно определить так, что он будет вызывать операции-зацепки (hooks) в определенных точках, разрешив тем самым расширение только в этих точках);
\end{itemize}

Шаблонные методы --- один из фундаментальных приемов повторного использования кода. Они играют особенно важную роль в библиотеках классов, поскольку предоставляют возможность вынести общее поведение в библиотечные классы. Шаблонные методы приводят к инвертированной структуре кода, которая подразумевает, что родительский класс вызывает операции подкласса, а не наоборот.

\pagebreak