\section*{Заключение [TODO]}
\addcontentsline{toc}{section}{Заключение [TODO]}

В рамках курсового проекта была разработана система извлечения многокомпонентных терминов и их переводных эквивалентов из параллельных научно-технических текстов.

Были рассмотрены и проанализированы существующие модели данных и области их применения. % возможность использования для решения поставленной задачи. 
С учётом требований к приложению была спроектирована база данных, а именно формализованы сущности и их связи, выбрана СУБД и реализован интерфейс для доступа к данным.

Разработанный веб-сервер позволяет получать на экране дисплея изображение полигональной модели, заданной пользователем. При разработке были учтены недостатки существующих программных решений для аппаратной платформы STM32.

В ходе выполнения экспериментально-исследовательской части работы было установлено, что разработанное программное обеспечение на тестовом оборудовании показывает высокую стабильность. При возрастании нагрузки на систему сохраняются эффективность работы программы и качество получаемого результата. Микроконтроллеры семейства STM32 подходят для решения относительно простых задач компьютерной графики, не требующих построения реалистических изображений с определением проекционных теней, использованием глобальной модели освещения и методов закраски, требующих больших вычислительных затрат.

% Было проведено исследование, которое показало, что применение кеширования на 80\% повышает производительность приложения.

В дальнейшем разработанное приложение будет развиваться в рамках микросервисной архитектуры. В частности, планируется добавить микросервисы для автоматической разметки текстов по описанному алгоритму и для перемещения данных из кеша в основное хранилище. Также планируется добавить графический интерфейс, позволяющий вручную размечать тексты и сохранять полученные данные при помощи разработанного веб-сервера.

%В рамках научно-исследовательской работы была проведена классификация сетевых операционных систем для устройств интернета вещей.
%
%В результате сравнения были выделены: 
%
%\begin{itemize}
%	\item Azure Sphere, Windows 10 IoT и Amazon FreeRTOS как наиболее функциональные и масштабируемые;
%	\item ОСРВ МАКС и KasperskyOS как наиболее доступные с точки зрения использования прикладных служб;
%	\item Ubuntu Core и Raspbian как наиболее адаптированные для бытового применения.
%\end{itemize}
%
%В ходе выполнения данной работы были решены следующие задачи.
%
%\begin{enumerate}[label*=\arabic*.]
%\item Проведён анализ предметной области интернета вещей;
%\item Рассмотрены существующие операционные системы для устройств интернета вещей;
%\item Сформулированы критерии сравнения и оценки рассмотренных операционных систем;
%\item Проведён сравнительный анализ существующих решений по выделенным критериям. 
%\end{enumerate}
%
%Таким образом, все поставленные задачи были выполнены, поставленная цель достигнута.

\pagebreak